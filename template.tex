\documentclass[11pt,a4paper,sans,english]{moderncv}        % possible options include font size ('10pt', '11pt' and '12pt'), paper size ('a4paper', 'letterpaper', 'a5paper', 'legalpaper', 'executivepaper' and 'landscape') and font family ('sans' and 'roman')
\moderncvstyle{casual}                             % style options are 'casual' (default), 'classic', 'oldstyle' and 'banking'
\moderncvcolor{blue}                               % color options 'blue' (default), 'orange', 'green', 'red', 'purple', 'grey' and 'black'
\nopagenumbers{}                                  % uncomment to suppress automatic page numbering for CVs longer than one page
\usepackage[utf8]{inputenc}                       % if you are not using xelatex ou lualatex, replace by the encoding you are using
\usepackage[scale=0.75,a4paper]{geometry}
\usepackage{babel}
%----------------------------------------------------------------------------------
%            personal data
%----------------------------------------------------------------------------------
\firstname{Jorge}
\familyname{Martínez}
%\title{Resumé title}                               % optional, remove/comment the line if not wanted
\address{street 8 number 4-211}{661020}{Santa Rosa de Cabal - Colombia}         % optional, remove/comment the line if not wanted; the "country" arguments can be omitted or provided empty
\mobile{+57 3217335404}                          % optional, remove/comment the line if not wanted
%\phone{phone number}                           % optional, remove/comment the line if not wanted
%\fax{fax number}                             % optional, remove/comment the line if not wanted
\email{jolumartinez@utp.edu.co}                               % optional, remove/comment the line if not wanted
\homepage{jolumartinez.com}                         % optional, remove/comment the line if not wanted
%\extrainfo{additional information}                 % optional, remove/comment the line if not wanted
\photo[64pt][0.4pt]{Jorge.jpg}                       % optional, uncomment the line if wanted; '64pt' is the height the picture must be resized to, 0.4pt is the thickness of the frame around it (put it to 0pt for no frame) and 'picture' is the name of the picture file
\quote{}                                 % optional, remove/comment the line if not wanted
%
\begin{document}
%-----       resume       ---------------------------------------------------------
\makecvtitle
\section{Education}
\cventry{2007--2015}{Electronic Engineer}{Universidad Tecnológica de Pereira}{Pereira}{}{}  % arguments 3 to 6 can be left empty
\cventry{2016--2018}{Master's degree in Electrical Engineering}{Universidad Tecnológica de Pereira}{Pereira}{}{}
\section{Master thesis}
\cvitem{Title}{Automatic synthesis of control strategies through temporal logics and Petri Nets for motion planning in autonomous systems}
\cvitem{Supervisors}{Ph.D Mauricio Holguín Londoño}
%\cvitem{description}{Short thesis abstract}
\section{Experience}
\subsection{Vocational}
\cventry{2016--2018}{Teacher}{Universidad Tecnológica de Pereira}{Pereira}{}{Hardware Design FPGA's, Industrial Automation, Introduction to Electronic Engineering, Electrical Maintenance.}
\cventry{2017--2018}{Researcher}{Universidad Tecnológica de Pereira}{Pereia}{}{Project: A Methodology to continuous synthesize finite states automatons through formal languages and temporal logics in Autonomous Ground Navigation Applications.}
\subsection{Miscellaneous}
\cventry{2013--2013}{Technical in Metrology of Electrical Variables}{Metrology Laboratory of Electrical Variables - UTP}{Pereira}{}{Electrical Equipment Calibration}
\cvitemwithcomment{ }{ }{}
\section{Languages}
\cvitemwithcomment{Spanish}{100$\%$}{Native}
\cvitemwithcomment{English}{Intermediate proficiency}{}
\cvitemwithcomment{ }{ }{}
%\cvitemwithcomment{Language 3}{Skill level}{Comment}

\section{Computer skills}
%\cvdoubleitem{Applications:}{Matlab, LabVIEW, ISE Design Xilinx, Android Studio, LyX, Microsoft Office Suite, G Suite.}{Operating Systems}{Linux, Windows (XP, Vista, 7, 8, 10), Android.}
\cvdoubleitem{Programming and markup Languages}{Python, Matlab, LabVIEW, VHDL, C, C++, PHP, SQL, LaTex.}{}{}
\section{Interests}
\cvitem{Ride Bicycle}{Exercise helps me keep fit.}
\cvitem{Play Piano}{It relaxes me and allows me to develop other skills.}
\cvitem{Travel}{I like to travel and to know other cultures}
\cvitem{Literature}{I enjoy reading classic literature}
\section{Other Skills}
\cvlistitem{Knowledge of research methodologies}
\cvlistitem{Computer vision}
\cvlistitem{Printed Circuit Design}
%\section{Extra 2}
%\cvlistdoubleitem{Item 1}{Item 4}
%\cvlistdoubleitem{Item 2}{Item 5}
%\cvlistdoubleitem{Item 3}{Item 6}
\section{References}
\begin{cvcolumns}
  \cvcolumn{Ph.D Mauricio Holguín (MsC Advisor)}{Universidad Tecnológica de Pereira}
  \cvcolumn{Ph.D Andrés Escobar Mejía}{Universidad Tecnológica de Pereira}
  \cvcolumn{Ph.D(C) Germán Andrés Holguín}{Universidad Tecnológica de Pereira}
\end{cvcolumns}
\clearpage
%-----       letter       ---------------------------------------------------------
% recipient data
\recipient{SINTEF\\Digital's Department of Mathematics and Cybernetics}{Trondheim}
\date{February 14, 2019}
\opening{Dear Sture Holmstrøm,}
\closing{Kind regards,}
%\enclosure{enclosures}          % use an optional argument to use a string other than "Enclosure", or redefine \enclname
\makelettertitle
My name is Jorge Luis Martinez, and I obtained my Master Degree in the electrical engineering master program at the Universidad Tecnológica de Pereira. I am writing this letter to express my motivation for the position "Researchers in estimation and robotics" at the SINTEF Digital's Department of Mathematics and Cybernetics.

I believe that my academic background is appropriated for the position at SINTEF. During the last two years, I have been researching about paths planning and motion planning problems with multiple agents, focused on the development of methodologies based on temporal logic in order to solve the combinatorial explode problem associated to the path planning. My thesis dissertation was "Automatic synthesis of control strategies through temporal logics and Petri Nets for motion planning in autonomous systems". I have a great interest in this field of research because I consider that autonomous driving robotic systems have great challenges of knowledge to be overcome.

I hold an Engineering Electronics degree in 2015, and from 2016 I started to be an assistant professor, teaching in several courses of electrical and electronics systems at the Universidad Tecnológica de Pereira. During my studies, academic and professional experience, I have been characterized by perseverance, dedication, and enthusiasm in the development of my work, execution of tasks and achieving the objectives. Always, my family have been an extra motivation to do my best in all the facets of my life.

I firmly believe that this position at SINTEF is an excellent chance to increase my knowledge, give contributions to the scientific community and pursue my dreams. I am confident that my academic record, experience, professional goals and my enthusiasm will make me a strong candidate for a place in this position. I would be honored if you decide to accept my application to research in your prestigious academic community.

I appreciate the time and efforts you invested in considering my application, and I am hoping to your prompt response.
\makeletterclosing
\clearpage

\recipient{SINTEF\\Digital's Department of Mathematics and Cybernetics}{Trondheim}
\date{February 14, 2019}
\opening{Dear Sture Holmstrøm,}
\closing{Kind regards,}
%\enclosure{enclosures}          % use an optional argument to use a string other than "Enclosure", or redefine \enclname
\makelettertitle

Following, are listed my most representative publications:


\begin{enumerate}
    \item \textbf{Title:} A Methodology for Movement Planning in Autonomous Systems with Multiple Agents.
    
    \textbf{Abstract:} In this paper, we present a methodology for the automatic synthesis of control strategies using temporal logics and Petri networks. Our method is applied to movement planning in autonomous systems with multiple agents. We propose a method to compute optimal routes for a team of multiple agents. Our proposed method is optimal with respect to the number of transitions executed by all team members. Linear Temporal Logic (LTL) is used as a language for task specification. Our method uses Petri Networks (PN) to model the multi-agent environment. Finally, we present an algorithm that implements the proposed method and its corresponding validation experiments.
    
    \textbf{Published in:} 2018 IEEE 2nd Colombian Conference on Robotics and Automation (CCRA).
    
    \textbf{Link:} https://ieeexplore-ieee-org.ezproxy.utp.edu.co/document/8588140
    
    \item \textbf{Title:} A methodology to evaluate combinatorial explosion using LTL in autonomous ground navigation applications.
    
    \textbf{Abstract:} In this work, the complexity of human-machine communication and problems related to autonomous systems in partially controlled environments is addressed. To this end, formal languages, state diagrams, relationships, autonomous and atomic propositions are used. The definition of logics, required to describe processes, which truth-values vary with time, to evaluate the need of language technologies for the performance of automatons in real environments involving human beings is considered. The importance of the investigation is presented and the proposed methodology that uses formal languages and temporal logics is implemented to synthesize automatons with finite states used in systems with state combinatorial explosion.
    
    \textbf{Published in:} 2017 IEEE 3rd Colombian Conference on Automatic Control (CCAC).
    
    \textbf{Link:} https://ieeexplore-ieee-org.ezproxy.utp.edu.co/document/8276430
    
    \item \textbf{Title:} A methodology to automaton synthesize in ground autonomous navigation systems control.
    
    \textbf{Abstract:} In this work, we propose a methodology for the synthesis of automatons in autonomous ground navigation systems with a global task. Our method is based on the design of automatons using regular grammars allowing for the generation of control policies for autonomous driving in a partially controlled environment where information is extracted using sensors. Furthermore, we showcase the problems that arise when approaching this problem with traditional synthesis of finite non-deterministic automatons. Finally, in the results section, we present the validation of the proposed method with simulations using MATLAB© and the Toolbox for Virtual Reality (V-Realm Builder).
    
    \textbf{Published in:} Scientia et technica, DOI: https://doi.org/10.22517/issn.2344-7214
    
    \textbf{Link:} http://revistas.utp.edu.co/index.php/revistaciencia/article/view/18121/13081
\end{enumerate}
\makeletterclosing
\end{document}
